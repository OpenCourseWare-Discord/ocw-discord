\documentclass{article}
% Required to support mathematical unicode
\usepackage[warnunknown, fasterrors, mathletters]{ucs}
\usepackage[utf8x]{inputenc}

% Always typeset math in display style
\everymath{\displaystyle}

% Use a larger font size
\usepackage[fontsize=14pt]{scrextend}

% Standard mathematical typesetting packages
\usepackage{amsfonts, amsthm, amsmath, amssymb}
\usepackage{mathtools}  % Extension to amsmath

% Symbol and utility packages
\usepackage{cancel, textcomp}
\usepackage[mathscr]{euscript}
\usepackage[nointegrals]{wasysym}

% Extras
\usepackage{physics}  % Lots of useful shortcuts and macros
\usepackage{tikz-cd}  % For drawing commutative diagrams easily
\usepackage{color}  % Add some colour to life
\usepackage{microtype}  % Minature font tweaks

% Common shortcuts
\def\mbb#1{\mathbb{#1}}
\def\mfk#1{\mathfrak{#1}}

\def\bN{\mbb{N}}
\def\bC{\mbb{C}}
\def\bR{\mbb{R}}
\def\bQ{\mbb{Q}}
\def\bZ{\mbb{Z}}

% Sometimes helpful macros
\newcommand{\func}[3]{#1\colon#2\to#3}
\newcommand{\vfunc}[5]{\func{#1}{#2}{#3},\quad#4\longmapsto#5}
\newcommand{\floor}[1]{\left\lfloor#1\right\rfloor}
\newcommand{\ceil}[1]{\left\lceil#1\right\rceil}

% Some standard theorem definitions
\newtheorem{Theorem}{Theorem}
\newtheorem{Proposition}{Theorem}
\newtheorem{Lemma}[Theorem]{Lemma}
\newtheorem{Corollary}[Theorem]{Corollary}

\theoremstyle{definition}
\newtheorem{Definition}[Theorem]{Definition}

\title{Problem Set 1 : Sets and Categories}
\date{14.08.2021}
\usepackage{calc}
\usepackage{accents}
\usepackage[margin=0.5in]{geometry}

\begin{document}
\maketitle
\subsection{Naive Set Theory}

1. Define a relation $\sim$ on the set $\mbb{R}$ of real numbers by setting $a \sim b \iff b-a \in \mbb{Z}$. Prove that this is an equivalence relation, and find a 'compelling' description for $\mbb{R}/\sim$. Do the same for relation $\approx$ on the plane $\mbb{R} \times \mbb{R}$ defined by declaring $(a_{1},a_{2}) \approx (b_{1},b_{2}) \iff b_{1}-a _{1} \in \mbb{Z}$ and $b_{2}-a_{2} \in \mbb{Z}$. [Aluffi Exercise 1.6]

\subsection{Functions on Sets}

2. Prove that the inverse of a bijection is a bijection and that the composition of two bijections is a bijection. [Aluffi Exercise 2.3]

\vspace{2.5mm}

\noindent
3. Show that if $A' \cong A''$ and $B' \cong B''$, and further $A' \cap B' = \varnothing$ and $A'' \cap B'' = \varnothing, then $A' \cup B' \cong A'' \cup B''$. Conclude that the operation $A \sqcup B$ is well-defined \textit{up to isomorphism} [Aluffi Exercise 2.9]

\subsection{Categories}

3. Let $C$ be a category. Consider a structure $C^{op}$ with 

\begin{itemize}
\item Obj($C^{op}$) := Obj(C); 
\item for A,B objects of $C^{op}$(hence objects of C), $Hom_{C^{op}}(A,B) := Hom_{C}(B,A)$. 

\end{itemize}

Show how to make this into a category. [Aluffi Exercise 3.1]

\vspace{2.5mm}
\noindent
4. Define a category $V$ by taking Obj($V$) = $\mbb{N}$ and letting $Hom_{V}(n.m)$ = the set of $m \times n$  matrices with real entires, for all $m.n \in \mbb{N}$. Use matrix multiplication to define composition. Does this category "feel" familiar? [Aluffi Exercise 3.6] 

\vspace{2.5mm}
\noindent 
5. Draw the relevant diagrams and define composition and identities for category $C^{A,B}$, mentioned in Example 3.9. Do the same for category $C^{\alpha,\beta}$ mentioned in Example 3.10. [Aluffi Exercise 3.11]

\subsection{Morphisms}
6. Let $A,B$ be objects of a category $C$, and let $f \in Hom_{C}(A,B)$ be a morphism. 
\begin{itemize}
\item Prove that if $f$ has a right-inverse, then $f$ is an epimorphism. 
\item Show that the converse doesn't hold, by giving an explicit example of a category and an epimorphism without a right inverse. 
\end{itemize}

[Aluffi Exercise 4.3]
\subsection{Universal Properties}
7. Show that in every category $C$ the produts $A \times B$ and $ B \times A$ are isomorphic, if they exist. [Aluffi Exercise 5.8]

\vspace{2.5mm}
\noindent
8. Let $C$ be a category with products. Find a reasonable candidate for the universal property that the product $A \times B \times C$ of \textit{three} objects of $C$ ought to satisfy,and prove that both $(A \times B) \times C$ and $A \times (B \times C)$ satisfy this universal property. Deduce that $(A \times B) \times C$ and $A \times (B \times C)$ are necessarily isomorphic. [Aluffi Exercise 5.9]


\end{document}
