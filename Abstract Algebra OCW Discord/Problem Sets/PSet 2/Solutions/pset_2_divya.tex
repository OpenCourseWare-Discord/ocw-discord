\documentclass{article}

%Required to support mathematical unicode
\usepackage[warnunknown, fasterrors, mathletters]{ucs}
\usepackage[utf8x]{inputenc}

% Always typeset math in display style
\everymath{\displaystyle}

% Use a larger font size
\usepackage[fontsize=14pt]{scrextend}

% Standard mathematical typesetting packages
\usepackage{amsfonts, amsthm, amsmath, amssymb}
\usepackage{mathtools}  % Extension to amsmath

% Symbol and utility packages
\usepackage{cancel, textcomp}
\usepackage[mathscr]{euscript}
\usepackage[nointegrals]{wasysym}

% Extras
\usepackage{physics}  % Lots of useful shortcuts and macros
\usepackage{tikz-cd}  % For drawing commutative diagrams easily
\usepackage{color}  % Add some colour to life
\usepackage{microtype}  % Minature font tweaks

% Common shortcuts
\def\mbb#1{\mathbb{#1}}
\def\mfk#1{\mathfrak{#1}}

\def\bN{\mbb{N}}
\def\bC{\mbb{C}}
\def\bR{\mbb{R}}
\def\bQ{\mbb{Q}}
\def\bZ{\mbb{Z}}

% Sometimes helpful macros
\newcommand{\func}[3]{#1\colon#2\to#3}
\newcommand{\vfunc}[5]{\func{#1}{#2}{#3},\quad#4\longmapsto#5}
\newcommand{\floor}[1]{\left\lfloor#1\right\rfloor}
\newcommand{\ceil}[1]{\left\lceil#1\right\rceil}

% Some standard theorem definitions
\newtheorem{Theorem}{Theorem}
\newtheorem{Proposition}{Theorem}
\newtheorem{Lemma}[Theorem]{Lemma}
\newtheorem{Corollary}[Theorem]{Corollary}

\theoremstyle{definition}
\newtheorem{Definition}[Theorem]{Definition}

%	PACKAGES AND OTHER DOCUMENT CONFIGURATIONS
%----------------------------------------------------------------------------------------

\usepackage{lastpage} % Required to determine the last page number for the footer

\usepackage{graphicx} % Required to insert images

\setlength\parindent{0pt} % Removes all indentation from paragraphs

\usepackage[most]{tcolorbox} % Required for boxes that split across pages

\usepackage{booktabs} % Required for better horizontal rules in tables

\usepackage{listings} % Required for insertion of code

\usepackage{etoolbox} % Required for if statements

%----------------------------------------------------------------------------------------
%	MARGINS
%----------------------------------------------------------------------------------------

\usepackage{geometry} % Required for adjusting page dimensions and margins

\geometry{
	paper=a4paper, % Change to letterpaper for US letter
	top=3cm, % Top margin
	bottom=3cm, % Bottom margin
	left=2.5cm, % Left margin
	right=2.5cm, % Right margin
	headheight=14pt, % Header height
	footskip=1.4cm, % Space from the bottom margin to the baseline of the footer
	headsep=1.2cm, % Space from the top margin to the baseline of the header
	%showframe, % Uncomment to show how the type block is set on the page
}


%----------------------------------------------------------------------------------------
%	HEADERS AND FOOTERS
%----------------------------------------------------------------------------------------

\usepackage{fancyhdr} % Required for customising headers and footers

\pagestyle{fancy} % Enable custom headers and footers

\lhead{\small\assignmentClass\ifdef{\assignmentClassInstructor}{\ (\assignmentClassInstructor):}{}\ \assignmentTitle} % Left header; output the instructor in brackets if one was set
\chead{} % Centre header
\rhead{\small\ifdef{\assignmentAuthorName}{\assignmentAuthorName}{\ifdef{\assignmentDueDate}{Due\ \assignmentDueDate}{}}} % Right header; output the author name if one was set, otherwise the due date if that was set

\lfoot{} % Left footer
\cfoot{\small Page\ \thepage\ of\ \pageref{LastPage}} % Centre footer
\rfoot{} % Right footer

\renewcommand\headrulewidth{0.5pt} % Thickness of the header rule

%----------------------------------------------------------------------------------------
%	MODIFY SECTION STYLES
%----------------------------------------------------------------------------------------

\usepackage{titlesec} % Required for modifying sections

%------------------------------------------------
% Section

\titleformat
{\section} % Section type being modified
[block] % Shape type, can be: hang, block, display, runin, leftmargin, rightmargin, drop, wrap, frame
{\Large\bfseries} % Format of the whole section
{\assignmentQuestionName~\thesection} % Format of the section label
{6pt} % Space between the title and label
{} % Code before the label

\titlespacing{\section}{0pt}{0.5\baselineskip}{0.5\baselineskip} % Spacing around section titles, the order is: left, before and after

%------------------------------------------------
% Subsection

\titleformat
{\subsection} % Section type being modified
[block] % Shape type, can be: hang, block, display, runin, leftmargin, rightmargin, drop, wrap, frame
{\itshape} % Format of the whole section
{(\alph{subsection})} % Format of the section label
{4pt} % Space between the title and label
{} % Code before the label

\titlespacing{\subsection}{0pt}{0.5\baselineskip}{0.5\baselineskip} % Spacing around section titles, the order is: left, before and after

\renewcommand\thesubsection{(\alph{subsection})}

%----------------------------------------------------------------------------------------
%	CUSTOM QUESTION COMMANDS/ENVIRONMENTS
%----------------------------------------------------------------------------------------

% Environment to be used for each question in the assignment
\newenvironment{question}{
	\vspace{0.5\baselineskip} % Whitespace before the question
	\section{} % Blank section title (e.g. just Question 2)
	\lfoot{\small\itshape\assignmentQuestionName~\thesection~continued on next page\ldots} % Set the left footer to state the question continues on the next page, this is reset to nothing if it doesn't (below)
}{
	\lfoot{} % Reset the left footer to nothing if the current question does not continue on the next page
}

%------------------------------------------------

% Environment for subquestions, takes 1 argument - the name of the section
\newenvironment{subquestion}[1]{
	\subsection{#1}
}{
}

%------------------------------------------------

% Command to print a question sentence
\newcommand{\questiontext}[1]{
	\textbf{#1}
	\vspace{0.5\baselineskip} % Whitespace afterwards
}

%------------------------------------------------

% Command to print a box that breaks across pages with the question answer
\newcommand{\answer}[1]{
	\begin{tcolorbox}[breakable, enhanced]
		#1
	\end{tcolorbox}
}

%------------------------------------------------

% Command to print a box that breaks across pages with the space for a student to answer
\newcommand{\answerbox}[1]{
	\begin{tcolorbox}[breakable, enhanced]
		\vphantom{L}\vspace{\numexpr #1-1\relax\baselineskip} % \vphantom{L} to provide a typesetting strut with a height for the line, \numexpr to subtract user input by 1 to make it 0-based as this command is
	\end{tcolorbox}
}

%------------------------------------------------

% Command to print an assignment section title to split an assignment into major parts
\newcommand{\assignmentSection}[1]{
	{
		\centering % Centre the section title
		\vspace{2\baselineskip} % Whitespace before the entire section title
		
		\rule{0.8\textwidth}{0.5pt} % Horizontal rule
		
		\vspace{0.75\baselineskip} % Whitespace before the section title
		{\LARGE \MakeUppercase{#1}} % Section title, forced to be uppercase
		
		\rule{0.8\textwidth}{0.5pt} % Horizontal rule
		
		\vspace{\baselineskip} % Whitespace after the entire section title
	}
}



\newcommand{\assignmentQuestionName}{Problem}
\newcommand{\assignmentTitle}{Problem Set 2}
\newcommand{\assignmentAuthorName}{Divya}
\newcommand{\assignemntClass}{Abstract Algbra}
\newcommand{\assignmentDueDate}{July 21, 2021}

\begin{document}

%Problem 1
\begin{question}
	\questiontext{Let $x$ and $y$ be two elements of a group $G$. Assume that each of the elements $x,y,xy$ has order 2. Prove that the set $H = \{ 1,x,y,xy \}$ is a subgroup of $G$, and it has order 4.}
	\answer{
		\begin{proof}
			\begin{itemize}
				\item We already have the identity $1 \in H$.
				\item As $x,y, xy$ are defined to be of order 2, thus they happen to be their own inverses. As $x^{2}=xx=1$, $y^{2}=yy=1$ and $(xy)^{2}=xyxy=1$. And $1$ is also the inverse of itself but it is \textit{not} of order 2. 
				\item Now we have to check that the multiplication is closed in $H$. One can make a Cayley multiplication table for this, but trivially, 
\[			\begin{aligned}
				1x =x1 =x \in H \\
				1y =y1 =y \in H \\
				xx = 1 \in H \\
				yy = 1 \in H \\
				xyy = x \in H \\
				xxy = y \in H \\
				1xy = xy1 =xy \in H
			\end{aligned}			\]
Thus, indeed the multiplication is closed in $H$. 

				\item And this is satisfied if we can show that all the elements of $H$ are distinct. Assume $x=1$ or $y=1$, then $xx = 1x$ must be 1 as x has an order of 2, but that's not the case since $1x=x$. Asssume $x=y$, then $yx=yy$ is contradictory since $yy=1$ and $x$ already is the inverse of itself. Now, for the last case, assume $x=xy$ or $y=xy$, now this is again impossible because if you multiply both the equations by $y$ and $x$ respectively, you get $xy=xyy$ whch is contradictory beause $xyy=x1=x$ and 1 is the unique identity. Similarly it's impossible for $xy=xxy$. Thus all the elements of the set are unique. Making $H \subset G$ a well-defined subgroup and $\lvert H \rvert = 4$.  
			\end{itemize}

		\end{proof}
	}

\end{question}

%Problem 2

\begin{question}
	\questiontext{Let $G$ be a group and let $a \in G$ have order $k$. If $p$ is a prime divisor of $k$, and if there is $x \in G$ with $x^{p}=a$, prove that $x$ has order $pk$.}
	\answer{
		\begin{proof}	

		\end{proof}
	}

\end{question}

%Problem 3
\begin{question}
	\questiontext{Let $G$ be a group in which $x^{2}=1$ for every $x \in G$, prove that $G$ must be abelian.}
	\answer{
	\begin{proof}
		As $x^{2}=1 \quad \forall x \in G$, thus $x$ is the inverse of itself. So for two distinct elements $x,y \in G \quad x \neq y$ we have, 
		\[ (xy)^{-1} = y^{-1}x^{-1} = yx \implies xyyx = 1 \] 
	So, $yx$ is the inverse of $xy$. But also as we have assosciativity under this group, $xyxy = xxyy= 1$, making $xy$ to be the inverse of itself, but as we know that in a well-defined groupo, the inverse of each element has to be unique so $yx = xy \quad \forall x,y \in G$, proving $G$ to be abelian. 
	\end{proof}
	}

\end{question}

%Problem 4
\begin{question}
	\questiontext{If $G$ is a group with an even number of elements, prove that the number of elements in $G$ of order $2$ is odd. In particular, $G$ must contain an element of order $2$.}
	\answer{
	\begin{proof}

	\end{proof} 
}
	

\end{question}

%Problem 5
\begin{question}
	\questiontext{Let $H$ be a subgroup of $G$ and $C(H) = \{ g \in G : gh = hg \quad \forall h \in H \}$ Prove that $C(H)$ is a subgroup of $G$. This is known as the \textit{centralizer} of $H$ in $G$.}
	\answer{
		\begin{proof}
			\begin{itemize}
				\item	Let $g_{1}, g_{2} \in C(H)$ with $g_{2}h_{1} = h_{2}g_{1}$ and $g_{2}h_{2}=h_{2}g_{2}$ for some $h_{1}, h_{2} \in H$. Multiplying these two equations we get, $g_{1}g_{2}h_{1}h_{2} = h_{1}h_{2}g_{1}g_{2} \implies g_{1}g_{2} \in C(H)$. Thus, the closure property is satisfied.   
				\item The identity in $G$ would be the identity in $H$, thus for $1 \in H \quad 1h = h1 \quad \forall h \in H$.. Thus $1 \in G$. So the identity is well-defined. 
				\item The inverses also exist, for $g \in C(H)$ we have $gh = hg$, for some $h \in H$. We have the inverse $g^{-1} \in H \subset G$. We first multiply this inverse on the left side, so we have $ghg^{-1} = hgg^{-1}$. Now we again multiply the inverse on the right side, so $g^{-1}ghg^{-1} = g^{-1}hgg^{-1} \implies hg^{-1} = g^{-1}h$. Thus $g^{-1} \in C(H)$. 
	\end{itemize}

		\end{proof}
	}

\end{question}

%Problem 6
\begin{question}
	\questiontext{Let $G$ b a group, let $X$ be a set, and let $f: G \to X$ be a bijection. Show that there is a unique operation on $X$, so that $X$ is a group and $f$ is an isomorphism.}
	\answer{
		\begin{proof}
			As $G$ is a group so we can define an operation on it such as : $m_{G} : G \times G \to G$. And similarly for $X$, $m_{X}: X \times X \to X$. And as we have $f : G \to X$ is a bijection thus it induces another bijection, namely : $f \times f : G \times G \to X \times X$. Thus,  
			\[
			\begin{tikzcd}
				G \times G \arrow[d, "m_{G}"'] \arrow[r, "f \times f"] && X \times X \arrow[d, "m_{X}"] \\
				G \arrow[r, "f"']                                      && X                            
			\end{tikzcd}
		\]

	Thus, for $X$ to be a group, $m_{G}$ has to be unique and $f$ would be a bijective group homomorphism, i.e an isomorphism. 
		\end{proof}
}
\end{question}

%Problem 7
\begin{question}
	\questiontext{Determine the center of $GL_{n}(\bR)$.}
	\answer{
		\begin{proof}



		\end{proof}
	}

\end{question}

%Problem 8 
\begin{question}
	\questiontext{Suppose given on $E$ an (associative and commutative) addition under which all the elements of $E$ are invertible and a multiplication which is non-associative but commutative and doubly distributive with respect to additoin. Suppose further that $n \in \bZ, n \neq0$ and $nx=0 \implies x=0$ in $E$. Show hat if, writing $[xy,y,z] = (xy)z-z(yz)$, the identity
	\[ [xy,u,z] + [yz,u,z] + [zx,u,y] = 0 \] 
holds, then $x^{m+n} = x^{m}x^{n}$ for all $x$ (show, by induction on $p$, the identity $[x^{q},y,x^{p-q}]=0$ holds for $1 \leq q < p$).}
	\answer{
		\begin{proof}




		\end{proof}
	}

\end{question}
\end{document}
