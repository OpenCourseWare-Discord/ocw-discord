\documentclass{article}
% Required to support mathematical unicode
\usepackage[warnunknown, fasterrors, mathletters]{ucs}
\usepackage[utf8x]{inputenc}

% Always typeset math in display style
\everymath{\displaystyle}

% Use a larger font size
\usepackage[fontsize=14pt]{scrextend}

% Standard mathematical typesetting packages
\usepackage{amsfonts, amsthm, amsmath, amssymb}
\usepackage{mathtools}  % Extension to amsmath

% Symbol and utility packages
\usepackage{cancel, textcomp}
\usepackage[mathscr]{euscript}
\usepackage[nointegrals]{wasysym}

% Extras
\usepackage{physics}  % Lots of useful shortcuts and macros
\usepackage{tikz-cd}  % For drawing commutative diagrams easily
\usepackage{color}  % Add some colour to life
\usepackage{microtype}  % Minature font tweaks
\usepackage[margin=0.5in]{geometry} 
\usepackage{calc}

% Common shortcuts
\def\mbb#1{\mathbb{#1}}
\def\mfk#1{\mathfrak{#1}}

\def\bN{\mbb{N}}
\def\bC{\mbb{C}}
\def\bR{\mbb{R}}
\def\bQ{\mbb{Q}}
\def\bZ{\mbb{Z}}

% Sometimes helpful macros
\newcommand{\func}[3]{#1\colon#2\to#3}
\newcommand{\vfunc}[5]{\func{#1}{#2}{#3},\quad#4\longmapsto#5}
\newcommand{\floor}[1]{\left\lfloor#1\right\rfloor}
\newcommand{\ceil}[1]{\left\lceil#1\right\rceil}
\newcommand{\ti}[1]{\textit{#1}}
% Some standard theorem definitions
\newtheorem{Theorem}{Theorem}
\newtheorem{Proposition}{Theorem}
\newtheorem{Lemma}[Theorem]{Lemma}
\newtheorem{Corollary}[Theorem]{Corollary}

\theoremstyle{definition}
\newtheorem{Definition}[Theorem]{Definition}

\title{Problem Set 2 : Group Homomorphisms and Isomorphisms}
\date{July 21, 2021}
\usepackage{calc}
\usepackage[margin=0.5in]{geometry}
\usepackage{hyperref}

\begin{document}
\maketitle
\noindent
\textbf{Referenced Texts :} \\
Artin's \textit{Algebra}: Chapter 2; Section 1-6. \\
Rotman's \textit{Advanced Modern Algebra} : Chapter 2; Section 1-3.\\
Rotman's \textit{Introduction to Theory of Groups}: Chapter 1. \\
Thomas Judson's \textit{Abstract Algebra: Theory and Applications}: Chapter 3; Section 1-4 \\
Nicholas Bourbaki's \textit{Algebre, Vol.I}: Chapter 1-3
	
	\vspace{5mm}

\par 
1. Let $x$ and $y$ be two elements of a group $G$. Assume that each of the elements $x,y$ and $xy$ has order 2. Prove that the set $H = \{1,x,y,xy\}$ is a subgroup of G, and it has order 4. \textbf{[Artin Exercise 4.7]}

\vspace{2.5mm}
\noindent

2. Let G be a group and let $a \in G$ have order k. If $p$ is a prime divisor of $k$, and if there is $x \in G$ with $x^{p} = a$, prove that x has order $pk$. \textbf{[Rotman, \textit{Advanced Modern Algebra} Exercise 2.24]}

\vspace{2.5mm}
\noindent

3. If $G$ is a group in which $x^{2}=1$ for every $x \in G$, prove that $G$ must be abelian. \textbf{[Rotman's \textit{Advanced Modern Algebra} Exercise 2.26]}

\vspace{2.5mm}
\noindent

4. If $G$ is a group with an even number of elements, prove that the number of elements in $G$ of order 2 is odd. In particular, $G$ must contain an element of order 2. \textbf{[Rotman, \textit{Advanced Modern Algebra}:  Exercise 2.27]}

\vspace{2.5mm}
\noindent

5. Let $H$ be a subgroup of $G$ and 
\[ C(H) = \{ g \in G : gh = hg \quad \forall h \in H \} \] 
Prove that $C(H)$ is a subgroup of $G$. This is known as the \textit{centralizer} of $H$ in $G$. \textbf{[Judson Exercise 3.4.53]}

\vspace{2.5mm}
\noindent

6. Let $G$ be a group, let $X$ be a set, and let $f: G \to X$ be a bijection. Show that there is a unique operation on $X$, so that $X$ is a group and $f$ is an isomorphism. \textbf{[Rotman, \textit{Intro to Theory of Groups}: Exercise 1.44]}

\vspace{2.5mm}
\noindent

7. Determine the center of $GL_{n}(\bR)$. \textbf{[Artin, Exercise 5.6]}

\vspace{2.5mm}
\noindent

8. [Optional] Suppose given on $E$ an (assosciative and commutative) addition under which all the elements of $E$ are invertive and a multiplication which is \textit{non-assosciative}, but commutative and doubly distributive with respect to addition. Suppose further that $n \in \bZ, n\neq 0$ and $nx=0 \implies x=0$ in $E$. Show that if, writing $[xy, y, z] = (xy)z - x(yz)$, the identity
\[ [xy,u,z] + [yz,u,z] + [zx,u,y] = 0 \]

holds, then $x^{m+n} = x^{m}x^{n}$ for all x (show, by induction on $p$, the identity $[x^{q},y, x^{p-q}]=0$ holds for $1 \leq q < p$). \textbf{[N. Bourbaki, \textit{Algebre, Vol.I}: Exercise 9, \S 3.]}
\end{document}
